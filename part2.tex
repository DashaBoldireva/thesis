\chapter{Результаты} \label{chapt2}

\section{Одиночное изображение} \label{sect2_1}

По анализу вариаьелниых участков V3-V4, V5-V6 гена 16S рРНК было установлено видовое разнообразие меромектических озер басейна Белого моря. Наиболее представленными оказались бактерии, относящиеся к следующим типам \textit{Chlorobi}, \textit{Proteobacteria}, \textit{Actinobacteria}.  Также было отмечено, что вариабельный участок V3-V4 хуже идентифицирует \textit{Cianobacteria}, чем вариабельный участок V5-V6. 

Для озера Трехцветное был построен бактериальный профиль (видовое разнообразие в зависимости от глубины забора образца). Данный профиль показал, что с увеличением глубины видовое разнообразие достаточно резко сокращается и увеличивается количество бактерий, относящихся к типу   \textit{Chlorobi}. Таким образом, мы можем сделать предположение о том, что на глубине около двух метров начинается зона хемоклина. Наше предположение подтверждается тем, что на данной глубине происходит резкое уменьшение концентрации кислорода и увеличение концентрации сероводорода. 

Повторное секвенирование гена 16S рРНК показало, что доминантным представителем является \textit{Chlorobium}. Последующий сбор ридов в контиги позволил лучше определить такснономическое разнообразие. Для более точного установления из данных по shoygun-секвенированию были выбраны риды, которые соответствуют вариабельным участкам V1, V5. 

По собранным контигам из shoygun-секвенирования были получены карты метаболических путей серы и азота. Таким образом, мы можем говорить о том, что данные бактерии являются азотофиксирующими, а также производят аноксигенный фотосинтез, используя серосодержащие вещества как  доноров серы.  

\clearpage
