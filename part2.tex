\chapter{Результаты} \label{chapt2}

\section{Секвенирование вариабельных участков V3-V4, V5-V6 гена 16S рРНК} \label{sect2_1}

Для каждого образца были получены следующие характеристики (см. таблицу \ref{tab:16Sread_Ill_characteristic})

\begin{table}[H]
\caption{Характеристики образцов, отсеквенированных на Illumina MiSeq.}\label{tab:16Sread_Ill_characteristic}
\begin{center}
\begin{tabular}{|c|p{0.15\linewidth}|p{0.15\linewidth}|p{0.15\linewidth}|p{0.15\linewidth}|}
\hline
Образец & Количество ридов участка V3-V4 & Количество ридов участка V5-V6 & Проверка качества ридов участка V3-V4 & Проверка качества ридов участка V5-V6 \\
\hline
10 КС 0,4 & 133 870 & 842 956 & 63,6 \% & 88,6 \% \\
30 ТЦ 1,9-4 & 189 853 & 914 442 & 72,1 \% & 91,0 \% \\
40 ЕЛ 3,0 & 159 276 & 883 811 & 72,7 \% & 90,9 \% \\
50 ТЦ 1,75 & 134 000 & 696 095 & 77,5 \% & 89,8 \% \\
60 КС 0,3 & 218 728 & 1 073 838 & 77,5 \% & 90,8 \% \\
80 ТЦ 1,9 & 220 225 & 1 304 206 & 73,9 \% & 90,3 \% \\
110 ЕР 2,5 & 225 734 & 1 306 371 & 76,5 \% & 91,0 \% \\
120 ЗМ 4,5 & 166 293 & 1 014 921 & 67,2 \% & 90,0 \% \\
130 ЕЛ 2,9 & 49 838 & 157 883 & 75,0 \% & 87,8 \% \\
\hline
\end{tabular}
\end{center}
\end{table}

По анализу вариабельных участков V3-V4, V5-V6 гена 16S рРНК было установлено видовое разнообразие меромектических озер басейна Белого моря. Наиболее представленными оказались бактерии, относящиеся к следующим типам \textit{Chlorobi}, \textit{Proteobacteria}, \textit{Actinobacteria}.  Также было отмечено, что вариабельный участок V3-V4 хуже идентифицирует \textit{Cianobacteria}, чем вариабельный участок V5-V6   (см. рис. \ref{ris:V3-V4, V5-V6}). 


\begin{figure}[h]
\begin{minipage}[h]{0.5\linewidth}
\center{\includegraphics[width=0.5\linewidth]{V3-V4} \\ а)}
\end{minipage}
\hfill
\begin{minipage}[h]{0.5\linewidth}
\center{\includegraphics[width=0.5\linewidth]{V5-V6} \\ б)}
\end{minipage}
\caption{Определение видового разнообразия по вареабельным участкам а) V3-V4, б) V5-V6}
\label{ris:V3-V4, V5-V6}
\end{figure}


Для озера Трехцветное был построен бактериальный профиль (видовое разнообразие в зависимости от глубины забора образца). Данный профиль показал, что с увеличением глубины видовое разнообразие достаточно резко сокращается и увеличивается количество бактерий, относящихся к типу   \textit{Chlorobi}. Таким образом, мы можем сделать предположение о том, что на глубине около двух метров начинается зона хемоклина. Наше предположение подтверждается тем, что на данной глубине происходит резкое уменьшение концентрации кислорода и увеличение концентрации сероводорода (см. рис. \ref{ris:concentr, 3cvet}).  

\begin{figure}[h]
\begin{minipage}[h]{0.5\linewidth}
\center{\includegraphics[width=0.5\linewidth]{concentr} \\ а)}
\end{minipage}
\hfill
\begin{minipage}[h]{0.5\linewidth}
\center{\includegraphics[width=0.5\linewidth]{3cvet} \\ б)}
\end{minipage}
\caption{а) Зависимость концентрация кислорода и сероводорода от глубины , б) Бактериальный профиль оз. Трехцветное}
\label{ris:concentr, 3cvet}
\end{figure}

\section{Секвенирование вариабельных участков V2-4-8/ V3-V6,V7-9 гена 16S рРНК} \label{sect2_2}


Повторное секвенирование гена 16S рРНК показало, что доминантным представителем является \textit{Chlorobium}. Последующий сбор ридов в контиги позволил лучше определить такснономическое разнообразие. Для более точного установления из данных по shotgun-секвенированию были выбраны риды, которые соответствуют вариабельным участкам V1, V5. 

По собранным контигам из shoygun-секвенирования была установлена метаболическая карта для \textit{Chlorobium}, что дало возможность анализировать и сравнивать данный тип бактерий с другими представителями рода (см. рис. \ref{img:kegg}). 


\begin{figure}[h]
  \includegraphics[width=10cm]{kegg}
  \centering
  \caption{Карта метаболических путей}
  \label{img:kegg}  
\end{figure}



Также для них были получены карты метаболических путей серы и азота. По метаболическому пути азота удалось установить, что данные бактерии являются азотофиксирующими. Метаболический путь серы указал на способность бактерий к аноксигенному фотосинтезу с использованием серосодержащих веществ как доноров серы. 

\center $\mathrm{C O_2 + 2H_2 S + h\nu = (C H_2 O) + H_2 O + 2 S}$


\clearpage
