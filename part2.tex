\chapter{Результаты} \label{chapt2}

\section{Секвенирование вариабельных участков V3-V4, V5-V6 гена 16S рРНК} \label{sect2_1}


Из полученных образцов было выделено ДНК в соответствии с методикой, описанной в разделе \ref{subsect1_2_1}. Для проверки качества полученной ДНК было проведено электрофоретическое разделение в агарозном геле. (картинка) Также была поставлена тестовая ПЦР для определения качества выделенной ДНК. (картинка?) 

Была приготовлена библиотека в соответствии с \textit{16S Metagenomic Sequencing Library Preparation }. Секвенирование проводилось на приборе Illumina MiSeq. В результате чего для каждого образца были получены следующие характеристики (см. таблицу \ref{tab:16Sread_Ill_characteristic})

\begin{table}[h]
\caption{Характеристики образцов, отсеквенированных на Illumina MiSeq.}\label{tab:16Sread_Ill_characteristic}
\begin{center}
\begin{tabular}{|c|p{0.15\linewidth}|p{0.15\linewidth}|p{0.15\linewidth}|p{0.15\linewidth}|}
\hline
Образец & Количество ридов участка V3-V4 & Количество ридов участка V5-V6 & Проверка качества ридов участка V3-V4, \% & Проверка качества ридов участка V5-V6, \% \\
\hline
10 КС 0,4 &                                                                                                                                                                                                                                                                                                                                                                                                                                                                                                                                                                        133 870 & 842 956 & 63,6 & 88,6  \\
30 ТЦ 1,9-4 & 189 853 & 914 442 & 72,1  & 91,0  \\
40 ЕЛ 3,0 & 159 276 & 883 811 & 72,7  & 90,9  \\
50 ТЦ 1,75 & 134 000 & 696 095 & 77,5  & 89,8  \\
60 КС 0,3 & 218 728 & 1 073 838 & 77,5  & 90,8  \\
80 ТЦ 1,9 & 220 225 & 1 304 206 & 73,9  & 90,3  \\
110 ЕР 2,5 & 225 734 & 1 306 371 & 76,5  & 91,0  \\
120 ЗМ 4,5 & 166 293 & 1 014 921 & 67,2  & 90,0  \\
130 ЕЛ 2,9 & 49 838 & 157 883 & 75,0  & 87,8  \\
\hline
\end{tabular}
\end{center}
\end{table}

Полученные риды были обработаны програмным обеспечением, встроенным в прибор, которое для классификации вариабельных участков гена 16S рРНК использует базу данных GreenGenes. После была проведена визуализация с использованием программы MEGAN5 (см. рис. \ref{ris:V3-V4, V5-V6}). 

\begin{figure}[h]
\begin{minipage}[h]{0.5\linewidth}
\center{\includegraphics[width=0.9\linewidth]{V3-V4} \\ а)}
\end{minipage}
\hfill
\begin{minipage}[h]{0.5\linewidth}
\center{\includegraphics[width=0.9\linewidth]{V5-V6} \\ б)}
\end{minipage}
\caption{Определение разнообразия до типа по вареабельным участкам а) V3-V4, б) V5-V6}
\label{ris:V3-V4, V5-V6}
\end{figure}

По анализу вариабельных участков V3-V4, V5-V6 гена 16S рРНК было установлено видовое разнообразие меромектических озер басейна Белого моря. Наиболее представленными оказались бактерии, относящиеся к следующим типам \textit{Chlorobi}, \textit{Proteobacteria}, \textit{Actinobacteria}.  Также было отмечено, что вариабельный участок V3-V4 хуже идентифицирует \textit{Cianobacteria}, чем вариабельный участок V5-V6   (см. рис. \ref{ris:V3-V4, V5-V6}). 

Также мы можем наблюдать родовое разнообразие для каждого образца (см. рис. \ref{ris:taxonomy_V3-V4, V5-V6}).

\begin{figure}[h]
\begin{minipage}[h]{0.5\linewidth}
\center{\includegraphics[width=0.9\linewidth]{taxonomy_profile_16S_Illum_V3-V4} \\ а)}
\end{minipage}
\hfill
\begin{minipage}[h]{0.5\linewidth}
\center{\includegraphics[width=0.8\linewidth]{taxonomy_profile_16S_Illum_V5-V6} \\ б)}
\end{minipage}
\caption{Определение родового разнообразия по вареабельным участкам для каждого образца а) V3-V4, б) V5-V6}
\label{ris:taxonomy_V3-V4, V5-V6}
\end{figure}

Нетрудно заметить, что в большинтсве случаев доминантными представителеми являются два рода бактерий, которые выделены рыжим и зеленым цветом, что относится к роду \textit{Chlorobium} и ... соответственно. (добавить про классификацию, что сейчас это уже один род) Немного другая картина для озера Кисло-Сладкое. Это связано с тем, что забор его образцов производился с небольшой глубины, то есть не достигнув зоны хемоклина. 
  
Для озера Трехцветное был построен бактериальный профиль (видовое разнообразие в зависимости от глубины забора образца), так как для него были получены образцы с различных глубин озера. Данный профиль показал, что с увеличением глубины видовое разнообразие достаточно резко сокращается и увеличивается количество бактерий, относящихся к типу   \textit{Chlorobi}. Таким образом, мы можем сделать предположение о том, что на глубине около двух метров начинается зона хемоклина. Наше предположение подтверждается тем, что на данной глубине происходит резкое уменьшение концентрации кислорода и увеличение концентрации сероводорода (см. рис. \ref{ris:concentr, 3cvet}).  

\begin{figure}[h]
\begin{minipage}[h]{0.5\linewidth}
\center{\includegraphics[width=0.7\linewidth]{concentr} \\ а)}
\end{minipage}
\hfill
\begin{minipage}[h]{0.5\linewidth}
\center{\includegraphics[width=0.9\linewidth]{3cvet} \\ б)}
\end{minipage}
\caption{а) Зависимость концентрация кислорода и сероводорода от глубины, б) Бактериальный профиль оз. Трехцветное}
\label{ris:concentr, 3cvet}
\end{figure}

\section{Shotgun-секвнирование}
После выделения ДНК (см. \ref{subsect1_2_1}) для каждого образца была измерена ее концентрация. Затем все образцы были смешаны в эквимолярном соотношении. После была приготовлена библиотека в соответствии с $IonXpress^{TM}$ \textit{Plus gDNA Fragment Library Preparation}. Секвенирование проводилось на приборе Ion Torrent PGM. В результате было получено 3 573 896 ридов, средняя длина которых 179 пар оснований. Далее полученные риды были собраны в контиги с использованием  сборщиков Newbler и SPAdes Genome Assembler, которые различаются алгоритмом сборки. Newbler использует метод сборки OLC (overlap-layout-consensus), SPAdes работает на основе графов де Брёйна. Использование двух разных методов сборки было необходимо для достижения лучшего результата определения принадлежности полученных ридов. Полученные контиги были проаннотированы с использованием программы PROKKA. Также были получены EC-номера для генов, которые были найдены в полученных контигах. По EC-номерам и с помощью онлайн ресурса KEGG были получены карты метаболических путей. 

По собранным контигам из shotgun-секвенирования была установлена метаболическая карта для \textit{Chlorobium}, что дало возможность анализировать и сравнивать данный тип бактерий с другими представителями рода (см. рис. \ref{img:kegg}). 

\begin{landscape}
\begin{figure}[h]
  \includegraphics[width=26cm]{kegg}
  \centering
  \caption{Карта метаболических путей}
  \label{img:kegg}  
\end{figure}
\end{landscape}

Также для них были получены карты метаболических путей серы и азота. По метаболическому пути азота удалось установить, что данные бактерии являются азотофиксирующими. Метаболический путь серы указал на способность бактерий к аноксигенному фотосинтезу с использованием серосодержащих веществ как доноров серы. 

\begin{center}
$\mathrm{C O_2 + 2H_2 S + h\nu \xrightarrow{} (C H_2 O) + H_2 O + 2 S}$
\end{center} 

\section{Секвенирование вариабельных участков V2-4-8/ V3-V6, V7-V9 гена 16S рРНК} \label{sect2_2}

Была приготовлена библиотека с использованием  Ion 16S Metagenomics Kit, в состав которого входит два набора праймеров, первый позволяет получать ампликоны вариабельных участков V2-4-8, второй - V3-V6, V7-V9. Секвенирование проводилось на приборе Ion Torrent PGM. В результате было получено 52 246 ридов, соответсвующих первому набору праймеров, и 50 043 ридов, соответствующих второму набору праймеров. После риды были собраны и проаннотированы с использованием RDP. Также были проаннотированые и не собранные в контиги риды. Затем все риды были визуализированы в MEGAN5 (см. рис. \ref{ris:sg+16S}). 

Из полученных при shotgun-секвенировании ридов были выделены риды, которые относятся к гену 16S рРНК. Они были также собраны, проаннотированы и добавлены к вышеописанным ридам . Затем все риды были визуализированы в MEGAN5 (см. рис. \ref{ris:sg+16S}). 

\begin{figure}[h]
\begin{minipage}[h]{0.6\linewidth}
\center{\includegraphics[width=0.9\linewidth]{sg+16S} \\ }
\end{minipage}
\hfill
\begin{minipage}[h]{0.5\linewidth}
\center{\includegraphics[width=0.4\linewidth]{legend} \\ }
\end{minipage}
\caption{Реконструкция бактериального профиля озера Трёхцветное}
\label{ris:sg+16S}
\end{figure}

В первом столбце рис. \ref{ris:sg+16S} представлены все риды, полученные при секвенировании гена 16S рРНК с помощью двух наборов праймеров. По нему мы можем определить доминантного представителя, как видно, это бактерии рода \textit{Chlorobium}. Во втором столбце представлены уже объединеные в контиги риды из первого столбца. Второй и последующие столбцы позволяют качественный, а не количественный состав исследуемого озера Трехцветное. Третий стобец представлет собой контиги гена 16S рРНК, которые были взяты из shotgun-секвенирования. По нему удается лучше установить биоразнообразие, так как в нем также содержатся и участки гена 16S рРНК, на которые не были подобраны праймеры. Наконец, четвертый столбец представлен контигами как с секвенирования гена 16S рРНК с помощью двух наборов праймеров, так и с shotgun-секвенирования. Таким образом, мы получили, что первый столбец в наибольшей степени показывает количественное соотношение организмов в озере, а четвертый - качественное. 

Такой подход позволил с хорошей точностью не только определить бактерии, которые болше остальных представлены в данном водоеме, но и других представителей. 


\clearpage
