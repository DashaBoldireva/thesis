Микроорганизмы отвечают за большинство биогеохимических циклов, которые оказывают влияние на атмосферу Земли и ее океанов. Пока еще данные организмы плохо исследованы, так как изучение и понимание метаболических путей микроорганизмов огрничивается отсутствием возможности получения чистых культур. В последних исследованиях ученые приступают к изучению бактерий из окружающей среды путем культуро-независимых методов: извлечение ДНК из образцов и трансформация ее в большие вставки клонов (large insert clones) . В настоящее время уже существует и продолжают появляться множество новых и различных методов по изучению микроорганизмов из окружающей среды. Самыми известными и популярными являются метагеномные подходы, которые дают наиболее информативные данные по видовому разнообразию сообщества микроорганизмов, а также его функциональным возможностям. Образцы для исследования по своей природе весьма разнообразны. На данный момент уже ислледованы метагеномы кишечника человека, почвы, водоемов...И на этом наука не останавливается и продолжает двигаться дальше.

\section{Меромектические водоемы}
Объектом наешго исследования был меромектический водоем - водоем, в котором практически
отсутствует циркуляция воды между слоями различной минерализации, разделенные хемоклином, вследствие чего вода нижнего слоя более минерализована и плотная, чем верхнего.
Меромиктические озера возникают по двум причинам. Во-первых, в тех случаях, когда породы, слагающие дно озера, являются растворимыми, например, гипсы. В этом случае они
частично растворяются в озерной воде, повышая минерализацию придонных слоев. Часто
бывает, что подземные воды, протекающие по растворимым породам, также становятся минерализованными, и поступают в глубинные слои в виде подводных ключей с повышенной
минерализацией воды. Во-вторых, меромиктия может возникать в результате опреснения
поверхностных слоев соленых водоемов, например, глубоко врезающихся в сушу морских
заливов, связанных с морем через относительно мелководные проливы. Пресные воды рек,
впадающих в такие заливы, частично замещают соленую морскую воду в поверхностных сло-
ях, тогда как в глубинных слоях соленость остается близкой к морской. К таким водоемам
относится Черное море. Это самый большой в мире меромиктический водоем.

В меромиктических водоемах нижние слои воды могут обладать значительно более высокой соленостью, чем поверхностные. Поверхностный менее соленый слой называется миксолимнионом. Нижний, более соленый – монимолимнионом. Интервал глубин, в котором происходит переход от миксолимниона к монимолимниону, т.е резко возрастает концентрация
солей, называется хемоклином.

Озерная вода, кроме солей, содержит в большем или меньшем количестве растворенные
газы: кислород, азот, углекислоту, метан, сероводород и водород. Водород, метан и сероводо-
род поступают в воду из иловых отложений в результате анаэробного (т.е., без присутствия
кислорода) распада органики, осуществляемого специфическими микроорганизмами.

Рассмтрим круговорот наиболее важных из этих элементов в стратифицированном водоеме. Кислород поступает из атмосферы, растворяясь в поверхностных слоях воды. Второй
источник кислорода – фотосинтез водорослей и высших растений, но об этом поговорим
позже, в следующих главах. Содержание кислорода в воде в значительной степени определяет очень важный показатель среды – окислительно-восстановительный потенциал (или
редокс-потенциал). Коротко можно сказать, что этот показатель говорит, какая обстановка
наблюдается в данной среде – окислительная (положительный редокс-потенциал) или восстановительная (отрицательный редокс- потенциал).В придонных слоях достаточно глубоких
водоемов фотосинтез невозможен, т.к водная толща поглощает почти весь свет. Следовательно, кислород может проникнуть в глубинные слои воды только за счет диффузии из верхних
слоев, либо в процессе перемешивания верхних слоев воды с нижними слоями.

Сероводород образуется сульфатредуцирующими бактериями в донных отложениях любых озер, т.к. даже в самом насыщенном кислородом озере с низкой продукцией органики
(так называемом олиготрофном озере) в иловых отложениях на глубине в несколько сантиметров кислорода уже нет. В таких озерах сероводород, произведенный в донных отложениях, окисляется кислородом уже в верхних слоях донных отложений, поэтому его практически
нет даже в придонных слоях водной толщи. В меромиктических озерах глубинные слои не
содержат кислорода, поэтому в таких озерах весь монимолимнион насыщается сероводородом.

Углекислый газ всегда содержится в природных водоемах как в виде \ce{CO2}
так и в виде
молекул угольной кислоты разной степени диссоциации: \ce{H2CO3}, \ce{HCO3^-} , \ce{CO3^2-} .
 Она потребляется фотосинтезирующими (так называемыми автотрофными организмами) в качестве
источника углерода для синтеза биомассы. Она же вновь образуется в результате разложения этой биомассы живыми организмами (так называемыми гетеротрофными организмами).
Таким образом осуществляется биотический круговорот углерода в водной экосистеме, как
и в биосфере в целом.

Живое население озер в основном представлено планктонными организмами. Планктон –
это все живые организмы, которые плавают в водной толще, но не способны противостоять
течению. Состав планктонных организмов чрезвычайно сложен, их большое разнообразие
позволяет лишь условно выделить основные группы.

Фитопланктон – к фитопланктону относятся фотосинтезирующие (фототрофные микроорганизмы). Это в основном одноклеточные зеленые, диатомовые водоросли, а также цианобактерии (сине-зеленые водоросли). Фитопланктон является автотрофным звеном водной
экосистемы, осуществляет первичный синтез органики, а все остальные группы планктона
потребляют эту органику, являясь участниками трофической цепи.

Бактериопланктон – к этому типу планктона относятся все основные группы бактерий,
взвешенные в водной толще и осуществляющие разные функции в водной 26 экосистеме.
Это питающиеся органическими веществами бактерии (гетеротрофные бактерии). Это также
хемо-автотрофные бактерии, строящие свою биомассу из углекислоты, так же как водоросли
и растения, но в отличие от них, получающие энергию не от солнечного света, а от реакций
окисления некоторых неорганических соединений, например \ce{H2S}, \ce{NH^4+} , \ce{NO^2-}.

Протозоопланктон – инфузории, жгутиковые и прочие крупные одноклеточные микроорганизмы, поедающие фито- и бактериопланктон.

Зоопланктон – многоклеточные существа, в основном рачки и коловратки, которые потребляют фито-, бактерио-, и протозоопланктон. Представители зоопланктона сильно различаются по размерам, от почти невидимых глазом коловраток и циклопов, до рачков-бокоплавов, которые достигают нескольких сантиметров в длину.

Рыбы – питаются фито- и зоопланктоном, а хищные рыбы являются верхним трофическим звеном, питаясь планктоядными рыбами. Молодь рыб питается также протозоопланктоном.

Кроме того, в экосистемах водоемов большую роль играют бентосные организмы, населяющие донные отложения. Это моллюски, черви, личинки насекомых. Разумеется, высшие
водные растения, развивающиеся в прибрежной полосе, также вносят значительный вклад
в первичную продукцию \cite{BioPhys10}.

\section{Методы исследования микробных сообществ}
Таким образом, мы видим, что несмотря на неоднородность минерализации в стратифицированных водоемах существует множество живых организмов. При изучении микробных
сообществ возникает вопрос, как мы можем их исследовать и описать? Обычно описание и
классифицирование микроорганизмов происходит по их фенотипу. Фенотип — широкое понятие, которое охватывает наблюдаемые характеристики организма, такие как морфология,
физиологическая активность, структура компонент клетки и иногда экологическая ниша,
которую занимает микроорганизм. Однако такие характеристики дают мало информации об
эволюционном родстве микроорганизмов, что должно быть основой классификационных систем. Более того большинство анализов фенотипичесих характеристик требуют клеточных
функций или роста, следовательно, для определения фенотипа необходимы чистые культуры, создание которых для многих видов организмов затруднено или невозможно. С другой
стороны, сравнение организмов для определения их эволюционного сродства может быть
достигнуто путем сравнения нуклеотидной последовательности геномов. Но такую информацию достаточно тяжело получить и интерпретировать. Также взаимосвязь между организмами может быть получена из сравнительного анализа последовательностей отдельных
генов в геноме \cite{Hugenholtz1996}.

\subsection{Секвенирование}
Достижения в такой области анализа (секвенирование ДНК) заключается в возможности изучения генетического разнообразия некультивируемых микроорганизмов. Например,
широко используемым методом для изучения разнообразия микробиоты является секвенирования ампликонов. В таком случае из образца выделяют ДНК, затем выделенную ДНК
амплифицируют методом ПЦР, далее производят секвенирование, после чего определяют,
какие микроорганизмы представлены в образце и в каком количественном соотношении. В
случае бактерий и архей вышеприведенным методом изучают малую рибосомальную субъединицу 16S, которая несет в себе как таксономическую, так и филогенетическиую информацию. Секвенирование ампликонов 16S рРНК выявило огромное количество микробиологического разнообразия и используется для определения характеристик биоразнообразия микроорганизмов в большом диапазоне окружающей среды, включающей кишечник человека,
термальные источники, антарктические минеральные почвы вулканов. Сравнения последовательностей 16S разных образцов показывает, как разнообразие микроорганизмов связано
с условиями окружающей среды. Но у данного метода есть свои ограничения и недостатки. Во-первых, можно неверно определить значительную долю разнообразия в сообществе
из-за различных ошибок, связанных с ПЦР. Во-вторых, секвенирование ампликонов может
совершить большой разброс оценок биоразнообразия. В дополнении, ошибки секвенирования и неправильно собранные риды могут выдать неверные последовательности, которые
будет тяжело определить. В-третьих, секвенирования ампликонов обычно дает только представление о таксономическом составе микрбиологического сообщества. Невозможно, точно
определить биологические функции сообщества, используя данный подход. Точность, с которой данный метод оценивает действительные функциональные различия сообществ, связана
с тем, насколько хорошо сообщестов представлено в существующих базах данных последовательностей геномов. Наконец, секвенирование ампликонов ограничивается анализом таксонов, для которых таксономически информативные генетические маркеры известны и могут
быть амплифицированны.

Однако существует альтернативный подход, с помощью которого можно изучить некультивируемые микроорганизмы и который помогает избежать предыдущие ограничения. Данный метод называется метод shotgun-секвенирования. Анологично секвенированию специфических участков генома, мы выделяем из клеток ДНК образца. Отличие состоит в том,
что впоследствии ДНК фрагментируется на небольшие участки определенной длины, которые затем
секвенируются. Это дает последовательности ДНК, которые выравниваются на различные
геномные участки для множества геномов, представленных в образце. Некоторые из полученных ридов будут отобраны из таксономически информативных участков генома, другие
— из кодирующих последовательностей, дающие представление о биологических функциях,
закодированных в геноме. В результате, данные метагеномики дают возможность одновременно изучать два аспекта микробиологических сообществ: \textit{кто это?} и \textit{какова его функция?}

Несмотря на преимущества данного метода перед другими, в метагеномном секвенированим также существуют некоторые проблемы. Во-первых, метагеномные данные достаточно
сложны и велики с точки зрения биоинформатического анализа. Например, определение генома, из которого были получены риды, может оказаться трудной задачей. Кроме того, большинство сообществ настолько разнообразны, что многие геномы не могут быть полностью
представлены ридами. В результате два рида одного и того же гена могут не перекрываться,
следовательно их невозможно напрямую сравнить, просто выравнивая последовательности.
Если риды перекрываются, то это еще не значит, что они относятся к одному геному или к
разным, которые могут выстраивать последовательность. Так же метагеномный анализ, как
правило, требует большой объем данных для определения значимых результатов из-за большого количества выбранной геномной информации, что может вызывать вычислительные
трудности. К счастью, современные вычислительные технологии развиваются достаточно
быстро, чтобы обойти данные проблемы. Во-вторых, метагеном может содержать нежелательную ДНК хозяина, особенно при изучении микробиоты. В некоторых ситуациях ДНК
хозяина может подавлять ДНК сообщества молекулярными методами, используемыми для
выборочного обогащения ДНК микроорганизмов до процесса секвенирования. В настоящее
время разрабатываются молекулярные и биоинформатические методы для фильтрации ДНК
хозяина до процесса сквенирования или после него. В-третьих, пока загрязнение является
общей проблемой для исследования окружающей среды, определение и удаление метагеномных последовательностей загрязняющих веществ особенно проблематично. Загрязнение метагенома может привести к неправильному анализу генетического разнообразия сообщества,
если количество загрязнений преобладает в исследуемом сообществе, особенно когда их очень
много или когда они имеют большой геном. Однако уже существуют программы, которые
могут справиться с данной проблемой. Наконец, создание метагенома дороже по сравнению
с секвенированием ампликонов, особенно в сложных сообществах или когда ДНК хозяина
значительно превосходит микробиологическую ДНК. К счастью, достижения в области ДНК
секвенирующих технологий повышают доступность метагеномного секвенирования.

Вышеприведенные проблемы, конечно, ограничивают применение метагеномных исследований, но благодаря упомянутым научным достижениям такой аналитический подход становится все более популярным для большинства лабораторий. За последние годы секвенирование метагенома было использовано для определения новых вирусов, установления харак-
теристик и функций некультивируемых бактерий, выявления новых и экологически важных
белков и идентификации таксономического и метаболического путей микробиоты кишечника
для здоровых и больных людей. Также метагеномный анализ может быть использован для
изучения микробиоты растений \cite{Sharpton2014}.

\subsection{Анализ на клеточном уровне}
Помимо вышеприеведнных методов иследования некультивируемых сообществ существуют другие не менее эффективные подходы для их изучения.

\subsubsection{Микроскопические методы исследования}
Так например, микроскопические методы исследования, которые используются непосредственно для визуализации нуклеиновых кислот. РНК-FISH (fluorescent \textit{in situ} hybridization)
флюоричцентно меченные олигонуклеотидные пробы для маркировки экспресированных молекул мРНК в фиксированных клетках. Пробы могут быть разработаны для любого гена с
известной последовательностью, инетерсующие клетки могут быть изучены вместе с окружающей их тканью, что немаловажно при исследовании тканей и опухолей. Например, методом
РНК-FISH были посчитаны единичные молекулы мРНК в единичных клетках и при помощи
стохастичекого моделирования cell-cell variability (? перевод) были получены биохимические
параметры, задействованные в транскрипции генов. Преимущество микроскопического метода изучения в том, что он сохраняет геометрию исследуемой ткани или клетки, что дает
возможность понять, как различные клеткаи или клеточные компоненты взаимодействуют
между собой и как они связаны с экспрессией генов. Ограничениями данного метода являются пропускная способность и возможность автоматизировать и распаралеллить геномные
измерения. Однако их можно избежать, разделив образцы по микрокамерам и используя
автоматическую микроскопию.

\subsubsection{Fluorescence-activated cell sorting}
Метод FACS (fluorescence-activated cell sorting) был разработан для определения характеристик единичных клеток. Суспензия клеток протекает через узкий поток жидкости. Клетки
проходят лазерный луч, и рассеянный на клетках свет попадает на детекторы, которые обрабатывают его и переводят в клеточные характеристики, такие как размер и зернистность.
После поток пропускается через сопло и разделяется на маленькие шарики, таким образом,
что в каждом шарике находится одна клетка. Шарики приобретают электрический заряд,
и затем электрически разделяют по различным емкостям, соответствующим определенным
клеточным характеристикам. FACS-приборы могут обрабатывать десятки тысяч клеток за
час и измерять приблизительно до 18 поверхностных маркеров одновременно. После анализирования клетки остаются жизнеспособными и могут быть использованы в биологических
экспериментах, например, трансплантация стволовых клеток. FACS становится основной технологией в биомедицине, особенно в установлении характеристик различных типов клеток
крови и имунной системы. Использование красителей, связывающихся с ДНК, позволяет
изучать содержение ДНК в единичных клетках. Что, с одной стороны, может быть исползовано для последующего каритипирования, а с другой стороны, для определения различий
между стадиями клеточных циклов. Основное ограничени данного метода заключается в
количестве различны флуорисцентных маркеров, которые могу быть измерены одновременно \cite{Kalisky2014}.

\subsubsection{Single-cell microarrays}
Одним из наиболее высокопроизводительных методов анализа является метод, исполюзующий микрочипы. ДНК микрочипы включают сотни тысяч ДНК фрагментов, расположенных на стеклянной подложке. Изначально микрочипы были разработаны для профилирования экспрессии генов. Впоследствии их стали использовать в изучении различных
аспектах микробной экологии, включащей круговорот метана, общее микробное разнообразие, диапазон биогеохимических функций. Первые олигонуклеотидные микрочипы были
применены для изучения микробной экологии, путем использования 9 проб 15-20-мерных
олигонуклеотидов для определения азотопотребляющих бактерий. В дальнейшем это позволило разработать высокочувствительные 16S рРНК ген-специфичные (Gene targeting ?)
микрочипы, которые могли анализировать около 30 000 20-мерных и 300 000 25-мерных олигонуклеотидных проб для исследования бактериального и архейного разнообразия. кДНК
чипы, изначально сделанные для установления таксономических взаимоотношений среди
вида Pseudomonas и использующие случайно созданные фрагменты генов длиной 1kb, позволили получать результаты с более высокой чувствительностью и лучшчим разрешением,
по сравнениюс с олигонуклеотидными микрочипами. Кроме того, CGAs (community genome
arrays) и GPMs (genome probing microarrays), которые используют микробные геномы, были
разработаны с использованием RSGP (reverse sample genome probing). Данный метода может
избежать ошибок ПЦР, а так же повысить их специфичность и эффективность. Методы CGA
и GPM используются для определения характеристик сложных микробных составов почвы,
рек, морских отложений, а так же ферментированных растительных кормов, контролируемые в течение процесса ферментации. Метагеномными микрочипами (MGA - metagenome
microarray) были созданы для бытрй характеризации метагеномных библиотек, содержащие полный геном микробиотического сообщества. От предыдущих данный вид микрочипов
отличается понятием зонда и мишени. Зонд, как правило, расположены на стеклянной подложке, где MGA обрабатывает содержащиеся на подложке мишени и использует специфически меченные гены как зонд. Использование обратного подхода совместно с метагеномными
микрочипами позволило быстро анализировать метагеномные библиотеки, которые сравнивают cosmid (? перевод) клоны, взятые из различных подземных вод, и fosmid (?) клоны
из морских отложений. Наконец, применение изотопных микрочипов, основанных на их возможности исползовать радиомеченные подложку, для идентфикации групп организмов сделало значительный прорыв в использовании микрочипов в микробной экологии, а так же
в физиологическом анализе различных микробных сообществ донных осадков (in activated
sludge). Как и все вышеописанные, данный метод имеет свои ограничения. Главными являются его стоимость и пропускная способность, а имеено количества образцов. Однако сейчас
стоимость микрочипового анализа резко падает, что дает ему возможность конкурировать
с методами, основаными на секвенировании, для анализа транскриптомов. Считается, что
микрочипы представляют более экономичный и более практичный метод для исследований,
требующих высокую пропускную способность.

\subsubsection{Single-cell genome sequencing}
Секвенирование геномов единичных клеток (single-cell genome sequencing) появилось как
новый способ, который сделал доступными для анализа большинство некультивируемых организмов. Даный метод был использован для получения генома TM7, кандидат филума с
экологической и клинической значимостью. (?) Также для исследований морских микробов,
насекомых симбионтов (?), организмов из сложных микробных сообществ, так как рубец
скота. Секвенирование единичных клеток используется также для сравнения генетических
последовательностей индивидуальных клеток, секвенируемых в популяции. Консорциум клеток, будь то популяция клеток ткани, микробное сообщество микробиоты человека или экологическе микроорганизмы, содержат геномные вариации, которые обеспечивают доступ к
биологически и экологически интересным процессам (?). Действительно, геномная неодно-
родность управляет некоторыми аспектами фенотипической неоднородности, наблюдаемыми
среди близкородственных клеток. Также геномный анализ единичных клеток используется
для установления происхождения опухолей и ткканей в многоклеточных организмах. Более того возможно изолировать и амплифицировать единичные хромосомы из единичных
клеток. Для этого разработан микрожидкостный прибор способный разделить и амплифицировать гомологичные копии каждой хромосомы из одной метафазной человеческой клетки
в независимых емкостях. Это позволило изучить две аллели каждой хромосомы независимо
друг от друга. Данный метод может быть использован для получения точной гаплоидной
генотипной информации каждой (personal ?) геномной последовательности, для понимания
мейотической рекомбинации и непосредственного изучения индивидуальных человеческих
лейкоцинтарных антигенов. Одним из основных ограничений амплфицированного секвенирования генома единичных клеток являются ошибки процесса амплификации, так как он
создает химеры и неспецифические продукты. Надеемся, что эти ограничения будут преодолимы в какой-то степени в скором будущем путем изобретения лучших методов амплификации и пространственной изоляции индивидуальных хромосом \cite{Kalisky2014, Roh2010}.

\section{Применение методов исследования}
Рассмотрим применение некоторых из методов подробнее.

Так, например, используя метод секвенирования гена 16S рРНК и денатурированного градиентного гель-электрофореза, было изучено прокариотическое разнообразие микробного мата арктического шельфа \cite{Bottos2008}. Изучение подобных экосистем предоставляет возможность изучить, как могла существовать жизнь, когда Земля переживала ледниковый период. Микробные экосистемы, существующие в условиях низких температур, также представляют интерес для астробиологов, так как известно, что полярные регионы Марса и спутники Юпитера и Сатурна содержат запасы замороженной воды способной поддерживать жизнь в прошлом или настоящем. Наконец, организмы, приспособленные к холоду,  и их ферменты все чаще используются в еде, химической и текстильной промышленности и для восстановления загрязненной окружающей среды, находящейся в холодных условиях. Подробное исследование экосистем, для которых характерны низкие температуры, привело к описанию микробных сообществ, проживающих в подобных условиях, включающих морские льды, озерные льды, слой вечной мерзлоты, ледниковые системы. Шельфовые ледники являются регионами припайя, которые существуют как многолетние экосистемы в Арктике и Антарктике. В обоих полярных регионах данная окружающая среда обеспечивает уникальное поведение для сложных сообществ микробного мата. Данные сообщества существуют замороженными и остаются нетронутыми внутри шельфовых ледников в течение многих лет и появляются в бассейнах талых вод, образующихся на вершине  шельфовых ледников в летние месяцы. Эти сообщества сталкивают с очень суровыми условиями окружающей среды, включающих постоянное воздействие очень низких температур, изменение концентрации солей и высокое воздействие ультрафиолетового излучения в течение летних месяцев. Целью данного исследования было оценить прокариотическое разнообразие, микробное распределение и метаболическую активность в микробных матах шельфовых ледников Ward Hunt и Markham, используя културо-зависимые и культуро-независимые методы. Изучение предоставило первое детальное описание прокариотических сообществ в микробных матах шельфовых ледников высоких широтах Арктики, первый молекулярный филогенетический анализ данного сообщества и первое описание метаболических путей данного сообщества при температуре много ниже нуля. 

Методом shotgun-секвенирования было исследована связь между биоразнообразием и потенциальной функциональной ролью в современных окружающих средах на примере стратифицированных озер  в карстовой области Banyoles, северо-восточной Испании \cite{MLO2015}. Связь между составом микробного сообщества и природными процессами, такими как круговорот углекислого газа, азота и серы  - одна из важнейших целей микробиологических экологов. Данная информация необходима нам для лучшего понимания структуры и функциональных возможностей микробных сообществ, фундаментальных механизмов, контролирующих микробные процессы и взаимодействия in situ, также для изучения взаимодействия между биосферой-гидросферой-геосферой. Однако понимание биологического взаимодействия в сложных системах достаточно затруднено.  Стратифицированные озера с аноксигенным сернистым дном – упрощенная система изучения для описания общих взаимодействий между биоразнообразием и биогеохимией, так как они имеют высокую активность, большую биомассу и низкое микробное разнообразие. Такие озера также могут потенциально быть смоделированы как современный аналог экосистемы океанов в течение долгого периода истории Земли. В данном примере проводилось изучение между металимнионом и хиполимнимнионом двух серосодержащих озер путем  shotgun-секвенирования метагенома и анализа некоторых метаболических путей in silico. После филогенетической и функциональной идентификации была исследована связь между составом микробного сообщества и его функциональностью в круговороте углерода, азота и серы. Из-за нехватки кислорода, большой микробной биомассы и большого вклада процесса фиксации CO2 в глубоких темных слоях было сделано предположение о высоком генетическом потенциале хемотрофной фиксации CO2 и тесной редокс-связи между круговоротами углерода, азота и серы. Функциональный анализ был сфокусирован на трех основных биогеохимических циклах для данного типа озер: круговорот углерода, азота и серы. В целом таксономическое распределение сообщества было оцененно при помощи филогенетической аннотации метагеномных ридов. Домен Bacteria численно преодбладает в генетическом составе микробного сообщества как аэробного-анаэробного взаимодействия, так и в анаэробном хиполимнионе. Более 95  процентов  всех таксономически выровненных метагеномных ридов соответсвуют бактериям с небольшим количесвтом представителей архей (более представлены в хиполимнионе, чем в металимнеоне), фагов и эукариот.  Секвенирование гена 16S рРНК подтвердило данно ераспределение. Наиболее представленными оказались бактерии вида \textit{Chlorobiales} (серо-зеленые бактерии), \textit{Campylobacterales} (эпсилон-протеобактерии), \textit{Burkholderiales} (бета-протеобактерии), \textit{Frankiales} (актинобактерии). Данные бактерии были сильно разделены между слоями и озерами и показали таксономическую кластеризацию в соответствии с редокс-потенциалом. По результатам данных показано, что гены, отвечающие за анаэробную фиксацию углерода, фиксацию азота и ассимиляторную сульфатредукцию, составляют значительный процент от всех проаннотированных ридов в хиполимнионе,  а гены для аэробного дыхания, ассимиляции азота и минерализации серы - в аэробном-анаэробном слое.  Таким образом, метагеномный подход предоставил возможность установить связь между бактериями и биогеохимическими циклами двух стратифицироанных озер, которые можно представить как современным аналогом древних океанов. 