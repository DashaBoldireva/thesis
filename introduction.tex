Объектом исследования был меромектический водоем - водоем, в котором практически
отсутствует циркуляция воды между слоями различной минерализации, разделенные хемо-
клином, вследствие чего вода нижнего слоя более минерализована и плотная, чем верхнего.
Меромиктические озера возникают по двум причинам. Во-первых, в тех случаях, когда по-
роды, слагающие дно озера, являются растворимыми, например, гипсы. В этом случае они
частично растворяются в озерной воде, повышая минерализацию придонных слоев. Часто
бывает, что подземные воды, протекающие по растворимым породам, также становятся ми-
нерализованными, и поступают в глубинные слои в виде подводных ключей с повышенной
минерализацией воды. Во-вторых, меромиктия может возникать в результате опреснения
поверхностных слоев соленых водоемов, например, глубоко врезающихся в сушу морских
заливов, связанных с морем через относительно мелководные проливы. Пресные воды рек,
впадающих в такие заливы, частично замещают соленую морскую воду в поверхностных сло-
ях, тогда как в глубинных слоях соленость остается близкой к морской. К таким водоемам
относится Черное море. Это самый большой в мире меромиктический водоем.
В меромиктических водоемах нижние слои воды могут обладать значительно более высо-
кой соленостью, чем поверхностные. Поверхностный менее соленый слой называется миксо-
лимнионом. Нижний, более соленый – монимолимнионом. Интервал глубин, в котором про-
исходит переход от миксолимниона к монимолимниону, т.е резко возрастает концентрация
солей, называется хемоклином.
Озерная вода, кроме солей, содержит в большем или меньшем количестве растворенные
газы: кислород, азот, углекислоту, метан, сероводород и водород. Водород, метан и сероводо-
род поступают в воду из иловых отложений в результате анаэробного (т.е., без присутствия
кислорода) распада органики, осуществляемого специфическими микроорганизмами.
Рассмтрим круговорот наиболее важных из этих элементов в стратифицированном водо-
еме. Кислород поступает из атмосферы, растворяясь в поверхностных слоях воды. Второй
источник кислорода – фотосинтез водорослей и высших растений, но об этом поговорим
позже, в следующих главах. Содержание кислорода в воде в значительной степени опре-
деляет очень важный показатель среды – окислительно-восстановительный потенциал (или
редокс-потенциал). Коротко можно сказать, что этот показатель говорит, какая обстановка
наблюдается в данной среде – окислительная (положительный редокс-потенциал) или вос-
становительная (отрицательный редокс- потенциал).В придонных слоях достаточно глубоких
4
водоемов фотосинтез невозможен, т.к водная толща поглощает почти весь свет. Следователь-
но, кислород может проникнуть в глубинные слои воды только за счет диффузии из верхних
слоев, либо в процессе перемешивания верхних слоев воды с нижними слоями.
Сероводород образуется сульфатредуцирующими бактериями в донных отложениях лю-
бых озер, т.к. даже в самом насыщенном кислородом озере с низкой продукцией органики
(так называемом олиготрофном озере) в иловых отложениях на глубине в несколько санти-
метров кислорода уже нет. В таких озерах сероводород, произведенный в донных отложени-
ях, окисляется кислородом уже в верхних слоях донных отложений, поэтому его практически
нет даже в придонных слоях водной толщи. В меромиктических озерах глубинные слои не
содержат кислорода, поэтому в таких озерах весь монимолимнион насыщается сероводоро-
дом.
Углекислый газ всегда содержится в природных водоемах как в виде CO2
так и в виде
молекул угольной кислоты разной степени диссоциации: H2CO3
, HCO –
3
, CO 2–
3 Она потреб-
ляется фотосинтезирующими (так называемыми автотрофными организмами) в качестве
источника углерода для синтеза биомассы. Она же вновь образуется в результате разложе-
ния этой биомассы живыми организмами (так называемыми гетеротрофными организмами).
Таким образом осуществляется биотический круговорот углерода в водной экосистеме, как
и в биосфере в целом.
Живое население озер в основном представлено планктонными организмами. Планктон –
это все живые организмы, которые плавают в водной толще, но не способны противостоять
течению. Состав планктонных организмов чрезвычайно сложен, их большое разнообразие
позволяет лишь условно выделить основные группы:
Фитопланктон – к фитопланктону относятся фотосинтезирующие (фототрофные мик-
роорганизмы). Это в основном одноклеточные зеленые, диатомовые водоросли, а также ци-
анобактерии (сине-зеленые водоросли). Фитопланктон является автотрофным звеном водной
экосистемы, осуществляет первичный синтез органики, а все остальные группы планктона
потребляют эту органику, являясь участниками трофической цепи.
Бактериопланктон – к этому типу планктона относятся все основные группы бактерий,
взвешенные в водной толще и осуществляющие разные функции в водной 26 экосистеме.
Это питающиеся органическими веществами бактерии (гетеротрофные бактерии). Это также
хемо-автотрофные бактерии, строящие свою биомассу из углекислоты, так же как водоросли
и растения, но в отличие от них, получающие энергию не от солнечного света, а от реакций
окисления некоторых неорганических соединений, например H2S, NH +
4
, NO –
2
.
Протозоопланктон – инфузории, жгутиковые и прочие крупные одноклеточные микро-
организмы, поедающие фито- и бактериопланктон.
Зоопланктон – многоклеточные существа, в основном рачки и коловратки, которые по-
требляют фито-, бактерио-, и протозоопланктон. Представители зоопланктона сильно раз-
личаются по размерам, от почти невидимых глазом коловраток и циклопов, до рачков-
бокоплавов, которые достигают нескольких сантиметров в длину.
5
Рыбы – питаются фито- и зоопланктоном, а хищные рыбы являются верхним трофиче-
ским звеном, питаясь планктоядными рыбами. Молодь рыб питается также протозоопланк-
тоном.
Кроме того, в экосистемах водоемов большую роль играют бентосные организмы, насе-
ляющие донные отложения. Это моллюски, черви, личинки насекомых. Разумеется, высшие
водные растения, развивающиеся в прибрежной полосе, также вносят значительный вклад
в первичную продукцию [1].
Таким образом, мы видим, что несмотря на неоднородность минерализации в стратифи-
цированных водоемах существует множество живых организмов. При изучении микробных
сообществ возникает вопрос, как мы можем их исследовать и описать? Обычно описание и
классифицирование микроорганизмов происходит по их фенотипу. Фенотип — широкое по-
нятие, которое охватывает наблюдаемые характеристики организма, такие как морфология,
физиологическая активность, структура компонент клетки и иногда экологическая ниша,
которую занимает микроорганизм. Однако такие характеристики дают мало информации об
эволюционном родстве микроорганизмов, что должно быть основой классификационных си-
стем. Более того большинство анализов фенотипичесих характеристик требуют клеточных
функций или роста, следовательно, для определения фенотипа необходимы чистые культу-
ры, создание которых для многих видов организмов затруднено или невозможно. С другой
стороны, сравнение организмов для определения их эволюционного сродства может быть
достигнуто путем сравнения нуклеотидной последовательности геномов. Но такую инфор-
мацию достаточно тяжело получить и интерпретировать. Также взаимосвязь между орга-
низмами может быть получена из сравнительного анализа последовательностей отдельных
генов в геноме [2].
Достижения в такой области анализа (секвенирование ДНК) заключается в возможно-
сти изучения генетического разнообразия некультивируемых микроорганизмов. Например,
широко используемым методом для изучения разнообразия микробиоты является секвени-
рования ампликонов. В таком случае из образца выделяют ДНК, затем выделенную ДНК
амплифицируют методом ПЦР, далее производят секвенирование, после чего определяют,
какие микроорганизмы представлены в образце и в каком количественном соотношении. В
случае бактерий и архей вышеприведенным методом изучают малую рибосомальную субъ-
единицу 16S, которая несет в себе как таксономическую, так и филогенетическиую информа-
цию. Секвенирование ампликонов 16S рРНК выявило огромное количество микробиологи-
ческого разнообразия и используется для определения характеристик биоразнообразия мик-
роорганизмов в большом диапазоне окружающей среды, включающей кишечник человека,
термальные источники, антарктические минеральные почвы вулканов. Сравнения последо-
вательностей 16S разных образцов показывает, как разнообразие микроорганизмов связано
с условиями окружающей среды. Но у данного метода есть свои ограничения и недостат-
ки. Во-первых, можно неверно определить значительную долю разнообразия в сообществе
из-за различных ошибок, связанных с ПЦР. Во-вторых, секвенирование ампликонов может
6
совершить большой разброс оценок биоразнообразия. В дополнении, ошибки секвенирова-
ния и неправильно собранные риды могут выдать неверные последовательности, которые
будет тяжело определить. В-третьих, секвенирования ампликонов обычно дает только пред-
ставление о таксономическом составе микрбиологического сообщества. Невозможно, точно
определить биологические функции сообщества, используя данный подход. Точность, с кото-
рой данный метод оценивает действительные функциональные различия сообществ, связана
с тем, насколько хорошо сообщестов представлено в существующих базах данных последова-
тельностей геномов. Наконец, секвенирование ампликонов ограничивается анализом таксо-
нов, для которых таксономически информативные генетические маркеры известны и могут
быть амплифицированны.
Однако существует альтернативный подход, с помощью которого можно изучить некуль-
тивируемые микроорганизмы и который помогает избежать предыдущие ограничения. Дан-
ный метод называется метод шотган секвенирования. Анологично секвенированию специ-
фических участков генома, мы выделяем из клеток ДНК образца. Отличие состоит в том,
что впоследствии ДНК режется на небольшие участки определенной длины, которые затем
секвенируются. Это дает последовательности ДНК, которые выравниваются на различные
геномные участки для множества геномов, представленных в образце. Некоторые из полу-
ченных ридов будут отобраны из таксономически информативных участков генома, другие
— из кодирующих последовательностей, дающие представление о биологических функциях,
закодированных в геноме. В результате, данные метагеномики дают возможность одновре-
менно изучать два аспекта микробиологических сообществ: кто это? и какова его функция?
Несмотря на преимущества данного метода перед другими, в метагеномном секвенирова-
ним также существуют некоторые проблемы. Во-первых, метагеномные данные достаточно
сложны и велики с точки зрения биоинформатического анализа. Например, определение ге-
нома, из которого были получены риды, может оказаться трудной задачей. Кроме того, боль-
шинство сообществ настолько разнообразны, что многие геномы не могут быть полностью
представлены ридами. В результате два рида одного и того же гена могут не перекрываться,
следовательно их невозможно напрямую сравнить, просто выравнивая последовательности.
Если риды перекрываются, то это еще не значит, что они относятся к одному геному или к
разным, которые могут выстраивать последовательность. Так же метагеномный анализ, как
правило, требует большой объем данных для определения значимых результатов из-за боль-
шого количества выбранной геномной информации, что может вызывать вычислительные
трудности. К счастью, современные вычислительные технологии развиваются достаточно
быстро, чтобы обойти данные проблемы. Во-вторых, метагеном может содержать нежела-
тельную ДНК хозяина, особенно при изучении микробиоты. В некоторых ситуациях ДНК
хозяина может подавлять ДНК сообщества молекулярными методами, используемыми для
выборочного обогащения ДНК микроорганизмов до процесса секвенирования. В настоящее
время разрабатываются молекулярные и биоинформатические методы для фильтрации ДНК
хозяина до процесса сквенирования или после него. В-третьих, пока загрязнение является
7
общей проблемой для исследования окружающей среды, определение и удаление метагеном-
ных последовательностей загрязняющих веществ особенно проблематично. Загрязнение ме-
тагенома может привести к неправильному анализу генетического разнообразия сообщества,
если количество загрязнений преобладает в исследуемом сообществе, особенно когда их очень
много или когда они имеют большой геном. Однако уже существуют программы, которые
могут справиться с данной проблемой. Наконец, создание метагенома дороже по сравнению
с секвенированием ампликонов, особенно в сложных сообществах или когда ДНК хозяина
значительно превосходит микробиологическую ДНК. К счастью, достижения в области ДНК
секвенирующих технологий повышают доступность метагеномного секвенирования.
Вышеприведенные проблемы, конечно, ограничивают применение метагеномных иссле-
дований, но благодаря упомянутым научным достижениям такой аналитический подход ста-
новится все более популярным для большинства лабораторий. За последние годы секвениро-
вание метагенома было использовано для определения новых вирусов, установления харак-
теристик и функций некультивируемых бактерий, выявления новых и экологически важных
белков и идентификации таксономического и метаболического путей микробиоты кишечника
для здоровых и больных людей. Также метагеномный анализ может быть использован для
изучения микробиоты растений [3].
Помимо вышеприеведнных методов иследования некультивируемых сообществ существу-
ют другие не менее эффективные подходы для их изучения.
Так например, микроскопические методы исследования, которые используются непосред-
ственно для визуализации нуклеиновых кислот. РНК-FISH (fluorescent in situ hybridization)
флюоричцентно меченные олигонуклеотидные пробы для маркировки экспресированных мо-
лекул мРНК в фиксированных клетках. Пробы могут быть разработаны для любого гена с
известной последовательностью, инетерсующие клетки могут быть изучены вместе с окржа-
ющей их тканью, что немаловажно при исследовании тканей и опухолей. Например, методом
РНК-FISH были посчитаны единичные молекулы мРНК в единичных клетках и при помощи
стохастичекого моделирования cell-cell variability (? перевод) были получены биохимические
параметры, задействованные в транскрипции генов. Преимущество микроскопического ме-
тода изучения в том, что он сохраняет геометрию исследуемой ткани или клетки, что дает
возможность понять, как различные клеткаи или клеточные компоненты взаимодействуют
между собой и как они связаны с экспрессией генов. Ограничениями данного метода явля-
ются пропускная способность и возможность автоматизировать и распаралеллить геномные
измерения. Однако их можно избежать, разделив образцы по микрокамерам и используя
автоматическую микроскопию.
Метод FACS (fluorescence-activated cell sorting) был разработан для определения характе-
ристик единичных клеток. Суспензия клеток протекает через узкий поток жидкости. Клетки
проходят лазерный луч, и рассеянный на клетках свет попадает на детекторы, которые об-
рабатывают его и переводят в клеточные характеристики, такие как размер и зернистность.
После поток пропускается через сопло и разделяется на маленькие шарики, таким образом,
8
что в каждом шарике находится одна клетка. Шарики приобретают электрический заряд,
и затем электрически разделяют по различным емкостям, соответствующим определенным
клеточным характеристикам. FACS-приборы могут обрабатывать десятки тысяч клеток за
час и измерять приблизительно до 18 поверхностных маркеров одновременно. После анали-
зирования клетки остаются жизнеспособными и могут быть использованы в биологических
экспериментах, например, трансплантация стволовых клеток. FACS становится основной тех-
нологией в биомедицине, особенно в установлении характеристик различных типов клеток
крови и имунной системы. Использование красителей, связывающихся с ДНК, позволяет
изучать содержение ДНК в единичных клетках. Что, с одной стороны, может быть испол-
зовано для последующего каритипирования, а с другой стороны, для определения различий
между стадиями клеточных циклов. Основное ограничени данного метода заключается в
количестве различны флуорисцентных маркеров, которые могу быть измерены одновремен-
но [4].
Одним из наиболее высокопроизводительных методов анализа является метод, исполю-
зующий микрочипы. ДНК микрочипы включают сотни тысяч ДНК фрагментов, располо-
женных на стеклянной подложке. Изначально микрочипы были разработаны для профи-
лирования экспрессии генов. Впоследствии их стали использовать в изучении различных
аспектах микробной экологии, включащей круговорот метана, общее микробное разнооб-
разие, диапазон биогеохимических функций. Первые олигонуклеотидные микрочипы были
применены для изучения микробной экологии, путем использования 9 проб 15-20-мерных
олигонуклеотидов для определения азотопотребляющих бактерий. В дальнейшем это поз-
волило разработать высокочувствительные 16S рРНК ген-специфичные (Gene targeting ?)
микрочипы, которые могли анализировать около 30 000 20-мерных и 300 000 25-мерных оли-
гонуклеотидных проб для исследования бактериального и архейного разнообразия. кДНК
чипы, изначально сделанные для установления таксономических взаимоотношений среди
вида Pseudomonas и использующие случайно созданные фрагменты генов длиной 1kb, поз-
волили получать результаты с более высокой чувствительностью и лучшчим разрешением,
по сравнениюс с олигонуклеотидными микрочипами. Кроме того, CGAs (community genome
arrays) и GPMs (genome probing microarrays), которые используют микробные геномы, были
разработаны с использованием RSGP (reverse sample genome probing). Данный метода может
избежать ошибок ПЦР, а так же повысить их специфичность и эффективность. Методы CGA
и GPM используются для определения характеристик сложных микробных составов почвы,
рек, морских отложений, а так же ферментированных растительных кормов, контролиру-
емые в течение процесса ферментации. Метагеномными микрочипами (MGA - metagenome
microarray) были созданы для бытрй характеризации метагеномных библиотек, содержа-
щие полный геном микробиотического сообщества. От предыдущих данный вид микрочипов
отличается понятием зонда и мишени. Зонд, как правило, расположены на стеклянной под-
ложке, где MGA обрабатывает содержащиеся на подложке мишени и использует специфиче-
ски меченные гены как зонд. Использование обратного подхода совместно с метагеномными
9
микрочипами позволило быстро анализировать метагеномные библиотеки, которые сравни-
вают cosmid (? перевод) клоны, взятые из различных подземных вод, и fosmid (?) клоны
из морских отложений. Наконец, применение изотопных микрочипов, основанных на их воз-
можности исползовать радиомеченные подложку, для идентфикации групп организмов сде-
лало значительный прорыв в использовании микрочипов в микробной экологии, а так же
в физиологическом анализе различных микробных сообществ донных осадков (in activated
sludge). Как и все вышеописанные, данный метод имеет свои ограничения. Главными явля-
ются его стоимость и пропускная способность, а имеено количества образцов. Однако сейчас
стоимость микрочипового анализа резко падает, что дает ему возможность конкурировать
с методами, основаными на секвенировании, для анализа транскриптомов. Считается, что
микрочипы представляют более экономичный и более практичный метод для исследований,
требующих высокую пропускную способность.
Секвенирование геномов единичных клеток (single-cell genome sequencing) появилось как
новый способ, который сделал доступными для анализа большинство некультивируемых ор-
ганизмов. Даный метод был использован для получения генома TM7, кандидат филума с
экологической и клинической значимостью. (?) Также для исследований морских микробов,
насекомых симбионтов (?), организмов из сложных микробных сообществ, так как рубец
скота. Секвенирование единичных клеток используется также для сравнения генетических
последовательностей индивидуальных клеток, секвенируемых в популяции. Консорциум кле-
ток, будь то популяция клеток ткани, микробное сообщество микробиоты человека или эко-
логическе микроорганизмы, содержат геномные вариации, которые обеспечивают доступ к
биологически и экологически интересным процессам (?). Действительно, геномная неодно-
родность управляет некоторыми аспектами фенотипической неоднородности, наблюдаемыми
среди близкородственных клеток. Также геномный анализ единичных клеток используется
для установления происхождения опухолей и ткканей в многоклеточных организмах. Бо-
лее того возможно изолировать и амплифицировать единичные хромосомы из единичных
клеток. Для этого разработан микрожидкостный прибор способный разделить и амплифи-
цировать гомологичные копии каждой хромосомы из одной метафазной человеческой клетки
в независимых емкостях. Это позволило изучить две аллели каждой хромосомы независимо
друг от друга. Данный метод может быть использован для получения точной гаплоидной
генотипной информации каждой (personal ?) геномной последовательности, для понимания
мейотической рекомбинации и непосредственного изучения индивидуальных человеческих
лейкоцинтарных антигенов. Одним из основных ограничений амплфицированного секвени-
рования генома единичных клеток являются ошибки процесса амплификации, так как он
создает химеры и неспецифические продукты. Надеемся, что эти ограничения будут преодо-
лимы в какой-то степени в скором будущем путем изобретения лучших методов амплифика-
ции и пространственной изоляции индивидуальных хромосом.
Рассмотрим применение некоторых из методов подробнее. [4, 5].